\chapter{Conclusion}

\begin{chapterAbstract}{Preview}
	This chapter concludes the thesis and summarizes the findings during this research work. It also provides the limitations of this work and points to new dimensions in which the current work can be enhanced. 
\end{chapterAbstract}
\clearpage

The development of the generic remotely tunable MT-FRHC based FLCS with VBCoA defuzzification in programmable hardware opens a line of approach to several explorations. This architecture provides a large number of functionalities to its users along with sufficient speed to drive most industrial processes like plasma position control in TFTR, water level control in coupled tank, etc. This system is standardized with MATLAB Fuzzy Logic Toolbox and has the ability to incorporate FIS files generated by this toolbox. The proposed systems are observed to perform well within the multiple testing paradigms mentioned in this work. Thorough investigations are done using multiple applications to ascertain generality and applicability of G-FLCS to various control applications.

\section{Summarized Results}
A brief overview of the various work done in the entire duration of the project is necessary to conclude this thesis. In chapter 2, MT-FRHC rule reduction technique is mathematically established and analyzed in comparison to the predominant FRHC technique. Both of the technique was implemented on a DSP hardware for timing analysis. MT-FRHC based G-FLCS achieved 27 \% higher performance in terms of speed compared to the OMF based G-FLCS. Without a code optimization based on memory and speed, the MT-FRHC based G-FLCS on DSP achieved a speed of around 10 KFLIPS. This chapter also introduced vertices based CoA computation technique and compared to Riemann sum based CoA computation. It was observed that on average, the proposed VBCoA consumed 5314 machine cycles in compared to 7640 machine cycles consumed by Riemann sum based CoA computation.

The system architecture for MT\hyp{}FRHC based G\hyp{}FLCS is presented in chapter 3. A WebUI is developed on ASP.NET and hosted using MS IIS7 for remote reconfigurability and tunability. A Genetic Algorithm based FCP extraction scheme was also described. The modules and the submodules supporting the proposed G-FLCS were explained in this chapter. This chapter does not provide any result ; however, it lays the essential ground work required for the actual implementation of the proposed G-FLCS. 

In chapter 4, the proposed MT-FRHC based G-FLCS with VBCoA defuzzification was implemented on a DSP hardware. The code was developed in C language and further optimized using linear ASM and intrinsic functions to achieve a 5 \% improvement on the code size. The proposed design was compared to existing designs that closely matches to the objective of this work. It was observed that proposed DSP based G-FLCS provided a speed of more than 13 KFLIPS in comparison to 5.5 KFLIPS \cite{Millan2008} and 11 KFLIPS \cite{Fu2010}.

\begin{table}[h!]
	\centering
	\caption{Comparison between Proposed hardware G\hyp{}FLCS and Similar Designs based of Reconfigurable Parameters}
	\label{my-label}
	\begin{tabular}{llll}
		\hline
		Year & Reference & Speed & Platform \\
		&  & (in FLIPS) &  \\ \hline
		\noalign{\vskip 2mm}
		2008 & Millan et. al.\cite{Millan2008} & 5.5 K & FPGA \\\hline
		\noalign{\vskip 2mm} 
		2010 & Yi Fu et. al.\cite{Fu2010} & 11 K & FPGA \\\hline
		\noalign{\vskip 2mm} 
		\multicolumn{2}{l}{Proposed G-FLCS} & 13 K & DSP \\ \hline
	\end{tabular}
\end{table}

%\begin{table}[h!]
%	\centering
%	\caption{Comparison between Proposed hardware G\hyp{}FLCS and Similar Designs based of Reconfigurable Parameters}
%	\label{tab:my-label}
%	\begin{tabular}{lllll}
%		\hline
%		Year & Reference & Speed & Platform & Features \\ 
%		&  & (in FLIPS) &  &  \\ \hline
%		2008 & Millan et. al.\cite{Millan2008} & 5.5 K & FPGA & \begin{tabular}[c]{@{}l@{}}Output MFs: Singleton (5)\\ I/O: 2-1\\ Input MFs: 8\\ Overlaps: Dynamic\\ Rules Evaluated : 64\end{tabular} \\ \hline
%		2010 & Yi Fu et. al.\cite{Fu2010} & 11 K & FPGA & \begin{tabular}[c]{@{}l@{}}Output MFs: (5)\\ I/O: 2-1\\ Input MFs: 5\\ Overlaps: 2\\ Rules Evaluated : 25\end{tabular} \\ \hline
%		\multicolumn{2}{l}{Proposed G-FLCS} & 13 K & DSP &  \begin{tabular}[c]{@{}l@{}}Output MFs: (7)\\ I/O: 4-1 (Configurable)\\ Input MFs: 7\\ Overlaps: Dynamic\\ (4)\\ Rules Evaluated : 49\\ (2401)\end{tabular} \\ \hline
%	\end{tabular}
%\end{table} 

Chapter 5 implements the proposed MT-FRHC based G-FLCS with VBCoA on C6748 DSP to control radial plasma position of Aditya TFTR. It was compared to an existing FLCS designed exclusively for this application \cite{Suratia2012}. Compared to the presently deployed control techniques, the proposed system achieved 59\% faster rise time and 87\% speedy settling time. However, a slight overshoot of 0.0009 m is reported by the proposed G-FLCS. This overshoot implies a plasma displacement of 9 mm in a vacuum chamber of 750 mm which is effectively 1.2 \% of the radius. 

\section{Contribution of this Thesis}
In summary, this research successfully contributes  
\begin{itemize}
	\item A G-FLCS module with WebUI has been proposed that can operate as standalone remotely tunable controller. All existing G-FLCS, had no user interactions \cite{Millan2008,Fu2010} and hence even though they were developed on field programmable hardware, the system architecture do not allow field programmability of the G-FLCS. The proposed MT-FRHC based GFLCS system can be programmed through the web interface. The novelty of this application lies in its system architecture which is elaborated in Chapter 3 and 4.
	\item A code optimization process is implemented to develop memory consumption and speed optimized G-FLCS controller on C6748 DSP processor. A 5\% memory saving was observed after this optimization process. Industry standard code optimization techniques are used in this work and they are explained in section 4.2.1. However, the standard HIL testing process involves high end debuggers and emulation devices. They also operate on various copyright protocols. However, in section 4.2.2, a naive UART based HIL testing process was described that provided performance and timing analysis of the proposed G-FLCS.   
	\item MT-FRHC rule reduction technique ensures that the proposed G-FLCS achieves an operating speed of around 10K FzLOPS.
	\item A GA based FCP extraction algorithm in conjunction with a Fuzzy PID approximation based initial FCP generation technique was proposed. Section 4.4.3 shows GA based FCP extraction technique to drive the G-FLCS for various control problems. The proposed method extracts minimal rules to reduce complexity. However, it does not guarantee a minimal number of overlaps among the input membership function. But, MT-FRHC algorithm reduces the complexity by dynamically controlling the overlaps. Thereby, together MT-FRHC and GA based G-FLCS provides a balanced performance that can be observed from section 5.7.
	\item A vertices based centroid computations for polygons were extensive used in geospatial applications \cite{Stankute2010}. In section 2.3.2, VBCoA defuzzification scheme is proposed. A novel algorithm for computation of vertices and its co-ordinates was proposed. This was observed to speed up the defuzzification process significantly compared to the widely used Reimann Integral Sum based CoA computation.
	\item Finally the G-FLCS is implemented to control radial plasma position of Aditya TFTR model. The observation obtained from this system was exciting as it provided 59\% faster rise time and 87\% speedy settling time in comparison to existing control schemes.
\end{itemize}


\section{Limitations of this Work}
The major limitations of the proposed MT-FRHC based G-FLCS with VBCoA design can be summarized as follows:
\begin{itemize}
	\item G-FLCS is developed on a programmable device. The methodology of G-FLCS is implemented using a DSP which works with a sequential code. A parallel architecture developed on FPGA or hybrid computing platform would unleash the complete power of the proposed method.
	\item G-FLCS is tested in HIL environment with Simulink models and not with practical systems. Although the HIL test results are promising, but a real-time testing is will assure that the system performs as expected. 
	\item G-FLCS is connected to a server PC using standard protocol. In this implementation the security of the data communication network is not stressed upon. It is indispensable to evaluate the network security and analyze the threats. 
	\item The proposed methods of GA based FCP extraction and Fuzzy PID approximation based initial FCP generation requires knowledge about the dynamics of a process plant. The actual essence of an FLCS is that it requires no knowledge about the dynamics of the plant. Although the basic MT-FRHC based G-FLCS with VBCoA operates on the ideology of a typical fuzzy system, the GA based FCP extraction and Fuzzy PID approximation based initial FCP generation is not applicable to process plants where the dynamics of the system is unknown.     
\end{itemize}

\section{Few Scope for Future Work}
The work presented in this thesis elaborates the design and implementation of a MT-FRHC based G-FLCS with VBCoA defuzzification method. This design has potential for wide explorations. Some of the significant area of future work includes the following.
\begin{itemize}
	\item The proposed G-FLCS architecture is implemented on the type-I Mamdani fuzzy logic control system. This architecture has been implemented using modular design methodology. The modules in proposed G-FLCS can be integrated with neural network to achieve a DSP based generic neuro-fuzzy	system. There are various applications based on neuro-fuzzy systems	\cite{Karuppusamy2015,Uddin2014}. These designs have been implemented on FPGA platform. As it has been already established that the proposed G-FLCS architecture can perform better than similar designs on FPGA. It will be interesting to analyse the performance of DSP based generic neuro-fuzzy system.
	
	\item An ASIP implementation of MT-FRHC based G-FLCS with VBCoA can be achieved. The proposed G-FLCS architecture is developed on a general	purpose DSP processor. The deign of G-FLCS includes many modules which have been implemented using linear ASM. There are many RISC based fuzzy processor reported in the literature \cite{Bosque2014b,Selvaperumal2014}. The proposed MT-FRHC based G-FLCS with VBCoA can be introduced in the instruction	set of these RISC based fuzzy processors. To realize this concept, MT-FRHC	rule-reduction process and VBCoA defuzzification method have	been implemented using linear ASM. However, the conversion of these	linear ASM functions into individual instructions and integrating these instructions into the existing instruction set can be a challenging aspect.
	
	\item This concept of G-FLCS can be extended to include Type-II Fuzzy sets. The Type-II Fuzzy sets have included uncertainty in membership functions that makes it more complex and challenging area of research. There	are various industrial application based on type-II FLCS \cite{Linda2011,Schrieber2015}. These designs can also be implemented in proposed G-FLCS architecture; by extrapolating the design reported in this thesis from type-I fuzzy sets to type-II fuzzy set. The resultant design can achieve a speed better than reported in these designs [93, 148]. It has already been shown in the thesis that the present G-FLCS architecture achieves the speed higher than similar designs implemented on FPGAs.
\end{itemize}

